\section{Einleitung}

Die Digitalisierung, der Prozess welcher analoge Daten ins digitale Format bringt, gibt es schon lange, wahrscheinlich seit der Geburt der Computer. Dies ist jedoch ein mühsamer Vorgang und beansprucht viel Zeit und muss teilweise auch von Hand bewältigt werden. Um die Digitalisierung  zu beschleunigen, gibt es nur eine Lösung - Automation. Für viele Anwendungen gibt es die auch schon. Was jeder kennt ist  die Spracherkennungen. Es gibt heutzutage kaum Smartphones, welche keinen Dienst wie Google Assistant, Siri, etc. zur verfügung stellen. Das Selbe gilt für die Digitalisierung von Büchern, was sich schon seit Jahren mit Hilfe von OCR Software (Optical Character Recognition) etabliert hat. Dennoch gibt es viele Nischengebiete, bei denen es an Angeboten fehlt.

In diesem Projekt lag der Fokus auf der Digitalisierung von OL-Karten. Diese werden zwar auf dem Computer generiert und dann Ausgedruckt. Man könnte also meinen, dass es zwecklos ist, die Karten wieder in die Welt der Nullen und Einer zu bringen, jedoch hat man als OL-Läufer den Zugang zu den digitalen Kartendaten oft nicht direkt. Generell können Gründe für die Digitalisierung über ein breites Spektrum variieren - sei es für die Erhaltung von Daten oder das vereinfachte Teilen über das Internet. Mein persönlicher Beweggrund war, dass ich als nicht perfekter OL-Läufer nach einem OL das Bedürfnis habe, meinen Lauf nochmals zu analysieren. Jedoch ist es nicht immer einfach, sich vorzustellen, was die beste Routenwahl gewesen wäre. Deshalb habe ich mich entschlossen, die Dinge selbst in die Hand zu nehmen, damit ich mir das Gelände besser unter die Lupe nehmen kann. Zu dem hat mich Informatik schon seit der Primarschule interessiert und ich habe auch schon mehrere Apps programmiert, welche momentan auf dem App Store zu finden sind.